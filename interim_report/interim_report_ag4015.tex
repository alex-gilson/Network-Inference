\documentclass{article}
\usepackage[utf8]{inputenc}
\usepackage{amsmath}
\usepackage{caption}
\usepackage{subcaption}
\usepackage{graphicx} %package to manage images
\graphicspath{ {Images/} }
\usepackage{xcolor}
\definecolor{dkgreen}{rgb}{0,0.6,0}
\definecolor{gray}{rgb}{0.5,0.5,0.5}
\definecolor{mauve}{rgb}{0.58,0,0.82}
\usepackage{listings}
\usepackage[T1]{fontenc} % Output font encoding for international characters
\usepackage[margin=1.5in]{geometry}


\usepackage{mathpazo} % Palatino font
% \usepackage[style=authoryear,backend=biber]{biblatex}
\lstset{frame=tb,
  language=Matlab,
  aboveskip=3mm,
  belowskip=3mm,
  showstringspaces=false,
  columns=flexible,
  basicstyle={\small\ttfamily},
  numbers=none,
  numberstyle=\tiny\color{gray},
  keywordstyle=\color{blue},
  commentstyle=\color{dkgreen},
  stringstyle=\color{mauve},
  breaklines=true,
  breakatwhitespace=true,
  tabsize=3
}

\usepackage[
    backend=biber,
    style=numeric,
		sorting=none,
    natbib=true,
    url=false,
    doi=true,
    eprint=false
]{biblatex}
\addbibresource{references.bib}
\begin{document}


\begin{titlepage} % Suppresses displaying the page number on the title page and the subsequent page counts as page 1
	\newcommand{\HRule}{\rule{\linewidth}{0.5mm}} % Defines a new command for horizontal lines, change thickness here
	
	\center % Centre everything on the page
	
	%------------------------------------------------
	%	Headings
	%------------------------------------------------
	
	\textsc{\LARGE Imperial College London}\\[1.5cm] % Main heading such as the name of your university/college
	
	\textsc{\Large Electrical and Electronic Engineering Department}\\[0.5cm] % Major heading such as course name
	
	%------------------------------------------------
	%	Title
	%------------------------------------------------
	
	\HRule\\[0.4cm]
	
	{\huge\bfseries Interim Report}\\[0.4cm] % Title of your document
	
	\HRule\\[1.5cm]
	
	%------------------------------------------------
	%	Author(s)
	%------------------------------------------------
	
	
	% If you don't want a supervisor, uncomment the two lines below and comment the code above
	{\large\textit{Project title}}\\
	\textsc{Structure and dynamics of large networks of interacting neurons} \quad \text{CID: 01112712} % Your name
	
	\vfill
	%------------------------------------------------
	%	Author(s)
	%------------------------------------------------
	
	
	% If you don't want a supervisor, uncomment the two lines below and comment the code above
	{\large\textit{Author}}\\
	\textsc{Alejandro Gilson Campillo} \quad \text{CID: 01112712} % Your name
	
	\vfill
	%------------------------------------------------
	%	Supervisor
	%------------------------------------------------
	
	
	{\large\textit{Supervisor}}\\
	\textsc{Prof. Pier Luigi Dragotti} 
	%------------------------------------------------
	%	Second Marker
	%------------------------------------------------
	
	\vfill
	
	{\large\textit{Second Marker}}\\
	\textsc{Dr. Wei Dai} 
	%------------------------------------------------
	%	Date
	%------------------------------------------------
	\vfill\vfill\vfill % Position the date 3/4 down the remaining page
	
	{\large\today} % Date, change the \today to a set date if you want to be precise
	
	%------------------------------------------------
	%	Logo
	%------------------------------------------------
	
	\vfill\vfill
	 
	\includegraphics[width=5cm]{imperialcollegelondon.jpg}\\[1cm] % Include a department/university logo - this will require the graphicx package	
	%----------------------------------------------------------------------------------------
	
	\vfill % Push the date up 1/4 of the remaining page
	
\end{titlepage}


\section{Introduction}

The brain is a complex machine, it allows the human being to think, communicate and feel. It does so thanks to the billions of neurons that communicate in a dense network through synapses. However, little is known about how it works. By studying how the neurons structure to to store and process information we can understand how the brain as a whole functions. This could have important applications in medicine for curing diseases such as Parkinson \cite{OldeDubbelinkKimT.E.2014Dbnt} and in machine learning for the development of more intelligent neural networks.

In order to infer the network structure of a set of neurons, these neurons are treated as a diffusion network where potential spikes travel from messenger to receptor just as if it were a disease. By evaluating the time of "infection", the relationship between two neurons can be probabilistically estimated. After computing the relationship between all of the neurons, an estimate of the topology of the network can be obtained. 

This project builds on previous research \cite{alexandru2018estimating, pranav_report}. In this published work, a biological neural network is simulated 

Here he approaches this problem using a certain Neuron model (Izhikevich) and a recently developed algorithm NetRate. A network of one type of neuron is generated and simulated and then the structure is inferred using NetRate. In this project he also suggests certain future work improvements. The aim of this project would be to improve on the methodologies used or look at the problem from another perspective. Some examples of this would be to optimize NetRate for parallel computing or including into the system different types of neurons.

The aim of this project is to improve on the state of the art research of network inference and the understanding of the underlying structure of the brain.

There is now available a dataset from 100 neurons from a mouse's brain. One of the aims of this project would also be to study how well the algorithm performs on real data and to be able to check its accuracy. 

\printbibliography 



\end{document}



