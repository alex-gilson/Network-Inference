\documentclass{article}
\usepackage[utf8]{inputenc}
\usepackage{amsmath}
\usepackage{caption}
\usepackage{subcaption}
\usepackage{graphicx} %package to manage images
\graphicspath{ {Images/} }
\usepackage{xcolor}
\definecolor{dkgreen}{rgb}{0,0.6,0}
\definecolor{gray}{rgb}{0.5,0.5,0.5}
\definecolor{mauve}{rgb}{0.58,0,0.82}
\usepackage{listings}
\usepackage[T1]{fontenc} % Output font encoding for international characters
\usepackage[margin=1.5in]{geometry}


\usepackage{mathpazo} % Palatino font
% \usepackage[style=authoryear,backend=biber]{biblatex}
\lstset{frame=tb,
  language=Matlab,
  aboveskip=3mm,
  belowskip=3mm,
  showstringspaces=false,
  columns=flexible,
  basicstyle={\small\ttfamily},
  numbers=none,
  numberstyle=\tiny\color{gray},
  keywordstyle=\color{blue},
  commentstyle=\color{dkgreen},
  stringstyle=\color{mauve},
  breaklines=true,
  breakatwhitespace=true,
  tabsize=3
}

\usepackage[
    backend=biber,
    style=numeric,
		sorting=none,
    natbib=true,
    url=false,
    doi=true,
    eprint=false
]{biblatex}
\addbibresource{references.bib}
\begin{document}


\begin{titlepage} % Suppresses displaying the page number on the title page and the subsequent page counts as page 1
	\newcommand{\HRule}{\rule{\linewidth}{0.5mm}} % Defines a new command for horizontal lines, change thickness here
	
	\center % Centre everything on the page
	
	%------------------------------------------------
	%	Headings
	%------------------------------------------------
	
	\textsc{\LARGE Imperial College London}\\[1.5cm] % Main heading such as the name of your university/college
	
	\textsc{\Large Electrical and Electronic Engineering Department}\\[0.5cm] % Major heading such as course name
	
	%------------------------------------------------
	%	Title
	%------------------------------------------------
	
	\HRule\\[0.4cm]
	
	{\huge\bfseries Interim Report}\\[0.4cm] % Title of your document
	
	\HRule\\[1.5cm]
	
	%------------------------------------------------
	%	Author(s)
	%------------------------------------------------
	
	
	% If you don't want a supervisor, uncomment the two lines below and comment the code above
	{\large\textit{Project title}}\\
	\textsc{Structure and dynamics of large networks of interacting neurons} \quad \text{CID: 01112712} % Your name
	
	\vfill
	%------------------------------------------------
	%	Author(s)
	%------------------------------------------------
	
	
	% If you don't want a supervisor, uncomment the two lines below and comment the code above
	{\large\textit{Author}}\\
	\textsc{Alejandro Gilson Campillo} \quad \text{CID: 01112712} % Your name
	
	\vfill
	%------------------------------------------------
	%	Supervisor
	%------------------------------------------------
	
	
	{\large\textit{Supervisor}}\\
	\textsc{Prof. Pier Luigi Dragotti} 
	%------------------------------------------------
	%	Second Marker
	%------------------------------------------------
	
	\vfill
	
	{\large\textit{Second Marker}}\\
	\textsc{Dr. Wei Dai} 
	%------------------------------------------------
	%	Date
	%------------------------------------------------
	\vfill\vfill\vfill % Position the date 3/4 down the remaining page
	
	{\large\today} % Date, change the \today to a set date if you want to be precise
	
	%------------------------------------------------
	%	Logo
	%------------------------------------------------
	
	\vfill\vfill
	 
	\includegraphics[width=5cm]{imperialcollegelondon.jpg}\\[1cm] % Include a department/university logo - this will require the graphicx package	
	%----------------------------------------------------------------------------------------
	
	\vfill % Push the date up 1/4 of the remaining page
	
\end{titlepage}


\section{Introduction}

The brain is a complex machine, it allows the human being to think, communicate and feel. It does so thanks to the billions of neurons that communicate in a dense network through synapses. However, little is known about how it works. By studying how the neurons structure to to store and process information we can understand how the brain as a whole functions. This could have important applications in medicine for curing diseases such as Parkinson \cite{OldeDubbelinkKimT.E.2014Dbnt} and epilepsy \cite{PONTEN2007918}, and in machine learning for the development of more intelligent neural networks.

In order to infer the network structure of a set of neurons, they are treated as a diffusion network where electrical spikes increase the likelihood of connected neurons to spike and therefore transmit a signal that travels as if it were a disease. By evaluating the time of "infection", the relationship between two neurons can be probabilistically estimated. After computing the relationship between all of the neurons, an estimate of the topology of the network can be obtained. 

Previous work on this topic \cite{pranav_report, alexandru2018estimating} evaluated the feasibility of using a maximum-likelihood estimator algorithm, NetRate \cite{rodriguez2011uncovering}, for the structure inference of biological neural networks. A network was simulated using the Izhikevic neuron model \cite{izhikevich2003simple} and the brian simulator \cite{10.3389/neuro.01.026.2009}. The connections between the neurons were then estimated, compared to the original network and the performance of the algorithm was evaluated.

The aim of this project is to improve on the state of the art research of network inference and the understanding of the underlying structure of the brain. There are many ways in which this can be done such as scalability and the addition of different types of neurons to the model.It would also be useful to test the accuracy of the algorithm on real interacting neurons. Developments in technology allow us to obtain spike data from individual neurons \cite{ito2016spontaneous, ito2014large, litke2004does}.


\section{Background}

\subsection{Definition of connectivity}

The definition of connectivity between neurons has a history of lack of consensus among the scientific community \cite{HORWITZ2003466}. Connectivity studies from different researches may lead to different results depending on how they define it, as they may be looking at different aspects of connectivity. The two main accepted definitions that are used are functional and effective connectivity.

Functional connectivity is the temporal correlation between spatially remote neurophysiological events \cite{friston1993functional}. Studies on this topic began with electroencephalography (EEG) measurements. Some methods to measure functional connectivity include the evaluation of the correlation in the frequency domain between EEG signals at different scalp locations \cite{pfurtscheller1999event}, and the cross-correlation of the time series measurements from a pair of electrodes \cite{gevins1985neurocognitive}. However, due to the volume conduction of brain tissue, the electrical activity from the scalp cannot infer the individual neuron behaviour below the electrode \cite{HORWITZ2003466}.

Effective connectivity was defined in \cite{friston1993functional} as the influence that one neural system exerts on another. Effective connectivity can be measured in terms of efficacy and contribution. At a synaptic level it can be expressed as in eq.\ref{eq:synaptic_effectivity}, where $x_{j}$ is the post-synaptic response to many pre-synaptic inputs $x_{i}$ and $\textbf{W}_{ij}$ is the efficacy of the connections between neurons $i$ and $j$. Contribution is reflected in eq.\ref{eq:synaptic_contribution} as the effect of $i$ on $j$ relative to all pre-synaptic inputs. Using this definition, directional effects are taken into account and a richer representation of the network can be attained. Following the approach in \cite{alexandru2018estimating}, this project will focus on the effective connectivity of neurons in a network.


\begin{equation}\label{eq:synaptic_effectivity}
x_{j} = \Sigma \textbf{W}_{ij}\times x_{i}
\end{equation}

\begin{equation}\label{eq:synaptic_contribution}
\frac{\textbf{W}_{ij}}{\Sigma \textbf{W}_{ij}}
\end{equation}

\subsection{Izhikevich neuron model}



\printbibliography 


\end{document}






