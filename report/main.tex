\documentclass[a4paper, twoside]{report}

\usepackage[english]{babel}
\usepackage[utf8]{inputenc}
\usepackage[T1]{fontenc}
\usepackage{listings}
\usepackage{hyperref}
\hypersetup{hidelinks}
\usepackage{lscape}
\usepackage{amsmath}
\usepackage{graphicx}
\usepackage[colorinlistoftodos]{todonotes}
\usepackage[
    backend=biber,
    style=numeric,
		sorting=none,
    natbib=true,
    url=false,
    doi=true,
    eprint=false
]{biblatex}
\addbibresource{bibs/sample.bib}

% \usepackage{caption}
\usepackage{subcaption}
% \usepackage{xcolor}
\usepackage{makeidx}
\definecolor{dkgreen}{rgb}{0,0.6,0}
\definecolor{gray}{rgb}{0.5,0.5,0.5}
\definecolor{mauve}{rgb}{0.58,0,0.82}
\usepackage{tikz}
\usetikzlibrary{arrows}
\usepackage{verbatim}
\usepackage{bm}
\usepackage{float}
\usepackage{mathpazo} % Palatino font
\usepackage[a4paper,top=3cm,bottom=4cm,left=3cm,right=3cm,marginparwidth=1.75cm]{geometry}
\usepackage[edges]{forest}
\definecolor{foldercolor}{RGB}{124,166,198}
\tikzset{pics/folder/.style={code={%
		\node[inner sep=0pt, minimum size=#1](-foldericon){};
		\node[folder style, inner sep=0pt, minimum width=0.3*#1, minimum height=0.6*#1, above right, xshift=0.05*#1] at (-foldericon.west){};
		\node[folder style, inner sep=0pt, minimum size=#1] at (-foldericon.center){};}
		},
		pics/folder/.default={20pt},
		folder style/.style={draw=foldercolor!80!black,top color=foldercolor!40,bottom color=foldercolor}
}

\forestset{is file/.style={edge path'/.expanded={%
		([xshift=\forestregister{folder indent}]!u.parent anchor) |- (.child anchor)},
		inner sep=1pt},
		this folder size/.style={edge path'/.expanded={%
		([xshift=\forestregister{folder indent}]!u.parent anchor) |- (.child anchor) pic[solid]{folder=#1}}, inner ysep=0.6*#1},
		folder tree indent/.style={before computing xy={l=#1}},
		folder icons/.style={folder, this folder size=#1, folder tree indent=3*#1},
		folder icons/.default={12pt},
}
\renewcommand{\baselinestretch}{1.3}
\lstset{frame=tb,
  language=Matlab,
  aboveskip=3mm,
  belowskip=3mm,
  showstringspaces=false,
  columns=flexible,
  basicstyle={\small\ttfamily},
  numbers=none,
  numberstyle=\tiny\color{gray},
  keywordstyle=\color{blue},
  commentstyle=\color{dkgreen},
  stringstyle=\color{mauve},
  breaklines=true,
  breakatwhitespace=true,
  tabsize=3
}

\title{Structure and dynamics of large networks of interacting neurons}
\author{Alejandro Gilson Campillo}
% Update supervisor and other title stuff in title/title.tex

\begin{document}
\begin{titlepage}

\newcommand{\HRule}{\rule{\linewidth}{0.5mm}} % Defines a new command for the horizontal lines, change thickness here

%----------------------------------------------------------------------------------------

\center % Center everything on the page

%----------------------------------------------------------------------------------------
%	HEADING SECTIONS
%----------------------------------------------------------------------------------------
\quad\\[1.5cm]
%\textsc{\LARGE MSc Thesis}\\[1.5cm] % Name of your university/college
\textsc{\Large Imperial College London}\\[0.5cm] % Major heading such as course name
\textsc{\large Department of Electrical and Electronic Engineering}\\[0.5cm] % Minor heading such as course title

%----------------------------------------------------------------------------------------
%	TITLE SECTION
%----------------------------------------------------------------------------------------
\makeatletter
\HRule \\[0.4cm]
{ \huge \bfseries \@title}\\[0.4cm] % Title of your document
\HRule \\[1.5cm]
 
%----------------------------------------------------------------------------------------
%	AUTHOR SECTION
%----------------------------------------------------------------------------------------

\begin{minipage}{0.4\textwidth}
\begin{flushleft} \large
\emph{Author:}\\
\@author\\ % Your name
\emph{CID:}\\
01112712\\
\end{flushleft}
\end{minipage}
~
\begin{minipage}{0.4\textwidth}
\begin{flushright} \large
\emph{Supervisor:} \\
Dr. Pier Luigi Dragotti \\
\emph{Second marker:} \\
Dr. Wei Dai
% Uncomment the following lines if there's a co-supervisor
%\\[1.2em] % Supervisor's Name
%\emph{Co-Supervisor:} \\
%Dr. Adam Smith % second marker's name
\end{flushright}
\end{minipage}\\[3cm]
\makeatother

%	LOGO SECTION
%----------------------------------------------------------------------------------------

\includegraphics[width=4cm]{title/imperialcollegelondon.jpg}\\[1cm] % Include a department/university logo - this will require the graphicx package
 
%----------------------------------------------------------------------------------------

%----------------------------------------------------------------------------------------
%	DATE SECTION
%----------------------------------------------------------------------------------------

{\large A thesis submitted for the degree of}\\[0.5cm]
{\large \emph{MEng Electrical and Electronic Engineering}}\\[0.5cm]
{\large \today}\\[2cm] % Date, change the \today to a set date if you want to be precise

\vfill % Fill the rest of the page with whitespace

\end{titlepage}


\begin{abstract}
This project tries to understand how real biological neural networks are connected by treating the system as a diffusion process. By selecting the connectivity weights of each of the neurons that maximize the likelihood of some recorded spikes occurring  the structure of the network is estimated. For this project, the speed and scalability of NetRate (the implemented algorithm) is improved and its constraints are analysed. Moreover, the model of neural network and inference algorithm are changed for it to be used on a real mouse’s spike data-set and a benchmark is presented for analysing inference accuracy when no ground truth is available.
\end{abstract}

\renewcommand{\abstractname}{Acknowledgements}
\begin{abstract}
It is usual to thank those individuals who have provided particularly useful assistance, technical or otherwise, during your project.
\end{abstract}

\tableofcontents
\listoffigures
\listoftables


\chapter{Introduction}

The brain is a complex machine, it allows the human being to think, communicate and feel. It does so thanks to the billions of neurons that communicate in a dense network through synapses. However, little is known about how it works. By studying how the neurons structure to store and process information we can understand how the brain as a whole functions. This could have important applications in medicine for curing diseases such as Parkinson \cite{OldeDubbelinkKimT.E.2014Dbnt} and epilepsy \cite{PONTEN2007918}, and in machine learning for the development of more intelligent neural networks.\\

In order to infer the network structure of a set of neurons, they are treated as a diffusion network where electrical spikes increase the likelihood of connected neurons to spike and therefore transmit a signal that travels as if it were a disease. By evaluating the time of "infection", the relationship between two neurons can be probabilistically estimated. After computing the relationship between all of the neurons, an estimate of the topology of the network can be obtained. \\

Previous work on this topic \cite{pranav_report, alexandru2018estimating} evaluated the feasibility of using a maximum-likelihood estimator algorithm, NetRate \cite{rodriguez2011uncovering}, for the inference of the structure of biological neural networks. A network was simulated using the Izhikevic neuron model \cite{izhikevich2003simple} and the Brian simulator \cite{10.3389/neuro.01.026.2009}. The connections between the neurons were then estimated, compared to the original network and the performance of the algorithm was evaluated. Recent developments in technology now allow scientists to obtain individual neuron spike information from the brain tissue \cite{ito2016spontaneous, ito2014large, litke2004does}. This data is very useful and serves as a mean of evaluating the performance of the algorithm with real neurons. Moreover, this information can help in creating simulated networks that resemble more the real biological ones.



\input{background/background.tex}

\graphicspath{ {project/} }
\chapter{Improving the speed of NetRate}

NetRate is a powerful algorithm that can make a good estimate of the connections of the nodes in a network. It analyses the spiking time of numerous neurons and constructs cascades that are used in the optimization problem. However, this is a computationally expensive process due to the large number of interactions between each of the neurons in the network. Moreover, as the size of the network increases, the number of cascades that are built grows exponentially. For a network of 10 neurons it only takes 8 minutes to obtain a result using one processor\footnote{The processors used throughout the whole work are the Intel(R) Core(TM) i7-4770 CPU @ 3.4GHz with 12GB of RAM}. However, for each addition of ten neurons to the system, the computation time increases threefold. \\

For this algorithm to eventually become useful in the area of neural signal processing it must be able to scale up and analyse systems of hundreds if not thousands of neurons. Fortunately, NetRate is an inherently parallel problem because it computes an independent optimization problem for each of the nodes of the network. A node \(j\) within a system of \(N\) nodes has \(N-1\) directed connections to all the neurons but itself. This makes the diagonal entries of the adjacency matrix equal to zero. Remember that the transmission rate of a node with itself is null \(\alpha_{j,i} = 0 \text{ if } j=i\). \\ 

A set of cascades is obtained from the spiking times of the system and assigned to the neuron node that originated them. This is necessary in order to compute each of the rows of the adjacency matrix. Then, they are used to build the components of the optimization problem. Therefore, after the cascade information is ready, each of the rows of the adjacency matrix can be computed with a different processor.\\

The objective function of NetRate's optimization problem makes use of logarithmic functions. Some solvers, such as SDPT3 \cite{toh1999sdpt3,tutuncu2003solving} (the one used by CVX) do not have support for these kind of problems and make use of recursive quadratic programming. This is a relatively new field of research \cite{powell1986recursive} where a quadratic approximation of the objective function is taken. The solution to the new problem will converge to the one of the original problem for a sufficient number of iterations at which the initialization values are shifted towards the solution of the previous iteration.\\

One of the problems encountered in \cite{pranav_report} was the lack of parallelization capabilities of the CVX software package used for NetRate. The package was not built to be parallelized and, therefore, aiming to do so would require a low level redevelopment of the software. However, the attempts of speed up were always carried out from within MATLAB. For this reason, a new approach, were the parallelization is achieved by opening several MATLAB instances is presented.

\subsection{Parallelization of NetRate}

When an algorithm is parallelized and each of the processes are completely independent, all the information required for its computation must be available from the beginning. Otherwise, a special communication protocol between the processes must be carried out. Then, after all of them have finished, their outputs must be recombined in the same way as if only one processor had computed the whole algorithm. An analysis of the necessary steps for computing NetRate is critical to understand what benefit can be obtained from parallelization:

\begin{enumerate}
\item The two files obtained from the Brian Simulator containing the indexes and times of each of the spikes must be converted into cascades and assigned to each of the neurons in the network that originated them.
\item The components that constitute the objective function and the constraints are constructed for each of the nodes in the network. This requires the characterization of the hazard and log survival functions.
\item Each of these components is assembled together to form the optimization problem in \ref{eq:optimization_netrate} for each of the rows of the adjacency matrix.
\item The software package CVX is used to compute the optimization problem that returns the optimal weights.
\item Post-processing of the solution is carried out. This includes cutting off adjacency weights below a certain threshold in order to promote sparsity.
\end{enumerate}

From the steps above, it can be observed that the one that requires the most amount of computation power is number 4, where CVX is executed. Moreover, the information required to compute each specific row of the adjacency matrix is obtained in step 2. Thus, the ramification of the jobs occurs from step 1 to 2. From this point onwards, the parallelization is possible. However, due to the insignificant computation time and an increased complexity of a parallelized step 2, it makes it unnecessary to parallelize. 
Steps 3 and 4 are very closely linked: in order to use CVX, the problem must be defined following the rules of CVX and, thus, it is easier if both of them are computed by the same processor.\\

It can be concluded that an optimal benefit from parallelization can be achieved by assigning the individual tasks corresponding to each of the rows of the adjacency matrix to the available number of processors. Let \(N\) be the number of nodes in a network, \(\bm{\alpha}_{n} = \{\alpha_{n,1},\alpha_{n,2},\cdots,\alpha_{n,N}\}\) and let \(\bm{C}_{n} \subset \bm{C}\text{, } n \in [1,N]\) be the set of cascades originated by node \(n\). Then, the structure of the proposed parallelized NetRate is described in figure \ref{fig:diagram_parallelization}.\\\\

\begin{figure}[H]
	\centering
	\includegraphics[trim={0 21cm 0 0}, width=\linewidth]{diagram_parallelization.pdf}
  \caption{Diagram of the NetRate parallelization process} 
	\label{fig:diagram_parallelization}
\end{figure}

Originally, the algorithm computed steps 3 and 4 sequentially. Once the transmission rates of a row of the adjacency matrix were computed, it continued with the following one. However, in a parallelized NetRate, they are all computed at the same time with step 5 also involving the stacking of each of the \(\bm{\hat{\alpha}}_{n}\) vectors to form the matrix \(\bm{\hat{A}}\).
The components in step 2 correspond to the ones seen in Eq.\ref{eq:solution_netrate} and whose solution is the assembled version of the one in step 3.\\

Once the structure of the algorithm is laid out, it still remains to plan how each of the jobs in steps 3 and 4 are will be assigned to the available processors. Not all the jobs take the same amount of time to be computed because they heavily depend on the number of cascades for their given node. As an illustration, a network of 10 neurons is simulated, and the number of cascades belonging to each node is shown in Eq.\ref{eq:num_cascades}. 

\begin{equation}
	\centering
	NoC = \{12, 21, 62, 166, 21, 67, 17, 30, 18, 81\}
	\label{eq:num_cascades}
\end{equation}

The mean of the distribution is 49 and the standard deviation 46. This means that the number of cascades differs significantly from node to node and that an appropriate way of distributing the jobs is required in order to keep all processors similarly busy. This is necessary because the whole algorithm will not finish until the last processor has estimated the weight parameters belonging to the last node. \\

Let \(M\) be the number of processors and \(N\) the number of nodes, where \(N > M\). The first \(M\) with the largest number of cascades are assigned in order to each of the processors. Each of the remaining \(N-M\) nodes is assigned to the processors following Eq.\ref{eq:argmin_processor}.

\begin{equation}
	\centering
	argmin_{n} \text{ } p_{i} + NoC_{n}, 
	\label{eq:argmin_processor}
\end{equation}

Where \(i \in [0,M]\) is the processor number, \(n \in [0, N - M]\) the node index and \(p_{i}\) is the sum of all the number of cascades assigned to processor \(i\). This method ensures that each of the processors has as close number of nodes as possible. Using Eq.\ref{eq:num_cascades} with \(N=10\), the distribution among \(M = 4\) processors becomes:

\begin{equation}
	\centering
	p_{1} = \{166\}, p_{2} = \{81, 21, 12\}, p_{3} = \{67, 21, 18\}, p_{4} = \{62, 30, 17\}
	\label{eq:distribution_processors}
\end{equation}

The resulting number of cascades for each of the processors becomes 166, 114, 106 and 109, respectively. This is a more balanced distribution than if the nodes had been assigned in any other way. \\

It remains to clarify in which order each of the nodes must be computed. One constraint that limits the algorithm is memory. The larger the number of cascades that need to be computed, the more memory is required for the algorithm to compute the weights. The relationship between the number of cascades and size of the network is exponential and, as it grows, NetRate makes use of a larger amount of memory. Each of the processors requires its own memory to perform NetRate in parallel. Thus, it becomes prohibitive to use several processors at the same time for large networks. However, some minor adjustments can be made to increase the capability of a parallelized NetRate by choosing which nodes are computed first. If all the largest nodes are computed at the same time the computer will not be able to finish the task for a sufficiently big network. For this reason, half of the processors will start with the nodes whose number of cascades is the lowest. The other half will do the opposite and compute the ones with the highest number of cascades. This way, the number of cascades computed at any given moment is levelled out and the likelihood of sudden spikes of memory usage are reduced. \\

As mentioned above, the CVX package cannot be parallelized naturally. Since the previous attempts to do so failed \cite{pranav_report}, a new method had to be implemented. For this project, instead of parallelizing NetRate from within MATLAB, several MATLAB instances are opened that work independently of each other. Each of these instances outputs a csv file and they are all combined by a Python script at the end of the computation. 

\subsection{Speed improvement results}

In this section the speed performance of the parallelized NetRate is evaluated. The computation time of NetRate for networks of different sizes, a stimulation period of 4000 ms and using 1, 4 and 8 processors is displayed in figure \ref{fig:speed_netrate}. Due to memory constraints of the computer, only up to 40 nodes were evaluated. As explained above, the more processors used at the time the more memory is required and this made the algorithm stall when using 8 processors in a network of 50 neurons.\\

\begin{figure}
\centering
\includegraphics[width=0.8\linewidth]{computation_time_netrate.pdf}
\caption{Computation time of NetRate}
\label{fig:speed_netrate}
\end{figure}

There is a significant improvement in the speed of the algorithm when comparing 1 processor to 4 and 8 processors. For 30 neurons, it takes 22, 8 and 9 minutes for 1, 4 and 8 processors whereas for 40 neurons, it takes 196, 77 and 78 minutes. Although these are good results, they are far from ideal. 
Firstly, the time it takes to do the computation with 4 processors is not 4 times faster than with just one processor. It is, in fact, approximately 2.6 for the networks with 30 and 40 neurons. 
Moreover, using 8 processors instead of 4 results in no speed improvement for any of the networks. It is difficult to understand this behaviour because deeper analysis shows that all 8 processors are busy during the computation of the algorithm. One possible explanation is that the only steps that are actually being parallelized are number 3 from figure \ref{fig:diagram_parallelization}, where the optimization problem in \ref{eq:optimization_netrate} for each of the rows of the adjacency matrix is built, and half of step 4, where the CVX package is used but the solver has not been called. This means that the solver in the CVX package is a shared resource among all the running MATLAB instances and that each of the processors waits in a queue to compute its own row of the adjacency matrix. Further investigation into this hypothesis leads in this direction: when printing the progress of the optimization solver SDPT3, there is a linear behaviour i.e., all processes print in an orderly fashion and there is no overlapping between them. However, further research into this issue must be carried out.\\

In this section it has been described what the steps of NetRate are, how the parallelization is implemented and what its limitations are. It is necessary to have a fast algorithm for it to be used with large networks. Finally, it was shown that although there is a significant improvement in the speed of NetRate, it is not as good as desired, and it also poses further restrictions on memory usage.  


\chapter{Simulating a biological neural network}

The inference of networks using NetRate has been used only on simulations. A random structure is generated and the spikes simulated using the Brian simulator. This is very useful because it allows the possibility of comparing the inferred network to a ground truth and evaluating the performance of the algorithm. However, the goal is to be able to implement it on real biological networks whose topology could provide insight to scientists. With this arise many difficulties that need to be dealt with.\\

It is important to find a suitable dataset to analyse. It must either be made out of voltage readings from an array of sensors in a cluster of neurons or spike times and indices\footnote{Here, the index is the neuron number that generates a specific spike} of those spikes. 

\section{Mouse cortical neuron dataset}
\section{Differences between the real and simulated biological neural network}
\section{Input stimulus to the system}
\subsection{System with random spikes}
\subsection{System with randomly spiking clusters}
\section{Cascade generation}
\subsection{Method of maximum cascades}
\subsection{Method of maximum independence}
\subsection{Optimal cascade generation}


\chapter{Inferring the connectivity of a mouse's neuronal system}
\section{Evaluating the performance without a ground truth}
\section{Results}

\input{evaluation/evaluation.tex}

\chapter{Software Package}

The final version of the Network Inference software package is fruit of the combined work from the authors in \cite{alexandru2018estimating} and myself. Every effort has been made in this project to deliver an algorithm that provides clear and comprehensive data to the scientist that intends to use it. Further changes to how the network is simulated and cascade generation modalities can be easily made with the recent changes to the structure of the software. A Github repository has been made available \href{https://github.com/gilson15/Network-Inference}{\underline{here}} for use and further development. 

\section{Programming languages}

The software that is used for this project is based on a set of scripts written in bash, Python and MATLAB. Each of these languages is used in different areas where they excel due to either availability of libraries or ease of use. \\

Bash scripts are used as the entry point to the software package, their main function is iterating through tests, calling the functions that simulate and infer networks, and move the output files in an organized manner.
MATLAB is the language in which the optimization problem is defined. The need for a convex optimization tool that implements recursive quadratic programming made it imperative to use this programming language. CVX is a software package that meets this criteria. Although there is also a version of this package available for Python, it does not make use of recursive quadratic programming and the accuracy of NetRate is, consequently, drastically reduced.\\

Python is the language of preference for this project. Not only is it easy to program and full of libraries, but it is also open-source and, therefore, easily available. A great effort has been made to move into this language from the previous version used in \cite{alexandru2018estimating}, which was mostly written in MATLAB. 
The parallelization process was carried out using this programming language with the use of the \textit{multiprocessing} library. Cascade generation and performance evalutation was translated from MATLAB to Python.\\

One of the major constraints that has to be taken into account when using NetRate is memory and computation power. This issue is accentuated when trying to infer networks with a large number of nodes or when the number of cascades is high. This issue can be dealt with by using large computers that have MATLAB and the CVX package installed. However, many times researches do not have this hardware capabilities at their hands. For this reason, the use of cloud computing services is critical for high performance data processing. However, many of these services do not offer MATLAB compatibility and the CVX package cannot, therefore, be used. A scalable network inference algorithm would need to be written in an open source language such as Python. One of the goals of this project has been to go in that direction. 




\section{Description of the algorithm}


Once the required components for the Network Inference software have been installed, the package is ready to be used. The entry point to the algorithm is \textit{runner.sh}. This is a bash script that defines the simulation variables, iterates through different testing options and calls \textit{main.sh}. This later file calls each of the modules of the algorithm that are in charge of simulating the network, generating cascades, parallelizing NetRate and obtaining results.\\

The first module that is executed is the \textit{izhikevichNetworkSimulation.py} script. It uses the Brian simulator to generate a network based on the equations in \ref{eq:izhikevich_ode}. It randomizes the connections between the neurons and applies a defined input stimulus to the system. Finally, the script outputs three csv files containing the network weights, the firing times and the indices of the neurons that fired.\\

After the network and spiking data is simulated, the next step is to generate the cascades. Given a cascade generation option, \textit{generate\_cascades.py} will produce these sets of spikes for NetRate to use. This script is also in charge of creating the \textit{A\_bad} and \textit{A\_potential} matrices. These contain the survival and hazard functions from step 2 in \ref{fig:diagram_parallelization}. Moreover, it also creates a vector containing the number of firings each of the neurons has in the set of cascades. \\

The script \textit{initial\_time.py} is a very simple program that starts measuring the time it takes for NetRate to infer the network. It stores the time in a pickle file that is latter opened in \textit{compare\_network.py}. This is used to test NetRate's parallelization performance.\\

The main module of the algorithm starts with \textit{parallelize\_cvx.py}. Given the number of processors, this python script is in charge of opening several MATLAB instances and telling each of them which nodes to compute NetRate on. This is done by calling several times (the number of processors) the function \textit{parallel\_cvx.m}. This function iterates through all the nodes assigned to its given processor and calls \textit{solve\_using\_cvx.m}, the actual NetRate function that defines the optimization problem using the \textit{A\_bad} and \textit{A\_potential} matrices, and the number of firings vector from \textit{generate\_cascades.py}.
Every time \textit{parallel\_cvx.m} has finished with one node, it outputs a \textit{csv} file containing the estimated weights from that node. All of these files are then rejoined in \textit{compare\_network.py} and compared to the ground truth. For this purpose, the MAE, accuracy, precision and recall performance metrics are used. The results are finally stored in a \textit{csv} file in its corresponding folder.


\section{Folder architecture}

Many files are used for this project. This is due to the large number of tasks that need to be carried out and that three different programming languages are used. Moreover, it is vital to keep track of all the output files from both the Brian Simulator and NetRate. Many different modalities of the algorithm can be run and they need to be stored in an organized manner. Moreover, once the simulation or the cascades have been generated, there is no need to recompute them again if they have been saved. In figure \ref{fig:folders} is displayed the tree structure of the software package.


\begin{figure}
\begin{forest}
	for tree={font=\sffamily, grow'=0,
	folder indent=.9em, folder icons,
	edge=densely dotted}
	[Network inference
		[r
			[network\_10\_nodes
				[network\_stimulation\_random\_spikes\_stimulation\_time\_100\_4	
				[cascades\_maximum\_cascades\_1.csv, is file]
				[inferred\_network\_1.csv, is file]
				[network\_1.csv, is file]
				[firings\_1.csv, is file]
				[indices\_1.csv, is file]
				[results\_1.csv, is file]]]
			[network\_20\_nodes]
			[network\_30\_nodes]]
		[Documentation]
		[runner.sh, is file]
		[main.sh, is file]
		[izhikevichNetworkSimulation.py, is file]
		[parallelize\_cvx.py, is file]
		[parallel\_cvx.m, is file]
		[solve\_using\_cvx.m, is file]
		[compare\_networks.py, is file]
		[generate\_cascades.py, is file]
		[initial\_time.py, is file]
	]
\end{forest}
\label{fig:folders}
\caption{Folder architecture for the Network Inference software package}
\end{figure}


\section{System requirements}

The Network Inference software has been run using a Linux operating system. Python's \textit{multiprocessing} library does not work on Windows. Moreover, the use of a linux terminal is required when using Windows because the algorithm makes use of bash scripts. Compatibility with MacOS can be achieved by translating these scripts. The necessary components of the algorithm can be found in table \ref{tab:requirements}.



\begin{table}[H]
\centering
\begin{tabular}{|l|l|}
\hline
Component  & Version \\ \hline
CVX				 & 2.1		 \\ \hline
numpy      & 1.15.4  \\ \hline
sympy      & 1.3     \\ \hline
jinja2     & 2.10    \\ \hline
brian2     & 2.2.1   \\ \hline
matplotlib & 1.5.3   \\ \hline
pandas     & 0.23.4  \\ \hline
networks   & 2.2     \\ \hline
\end{tabular}
\caption{System requirements for the network inference software package}
\label{tab:requirements}
\end{table}









\chapter{Conclusion and future work}

\section{Summary of achievements}

The objective of this project was to improve the state of the art of neural network inference by increasing NetRate's performance or scalability. It was also intended for this algorithm to be used for the network inference of a biological neural networks. This objective brought along many challenges that had to be dealt with beforehand such as changing the network model and updating the cascade generation method. Overall, this project has been successful at achieving the defined goals. \\

\begin{enumerate}
\item In chapter 1, a brief explanation of the project was given. It also included what the objectives were. 
\item Chapter 2 explained the mathematical background behind the network inference algorithm and the previous work done in this area.
\item Chapter 3 explained how a novel method could achieve a certain level of parallelization in NetRate and increase the speed of the algorithm.
\item In chapter 4, the model for a biological neural network that mimicked a real biological one was devised. Moreover, two new cascade generation methods were proposed in order to adapt to the changes in the network model.
\item A new performance metric was proposed in chapter 5 for networks without a ground truth. Its drawbacks were analysed and illustrative results were shown.
\item In chapter 6 the suitability of the changes in the algorithm done in chapter 4 is proved. Then a biological neural network is inferred and its characteristics are analysed. Moreover, the performance metric outlined in chapter 5 is used to test the knowledge of the network. It was shown that NetRate is able to capture some network information from the real data.
\item Finally, in chapter 7, the software package used for this project is described. Moreover, the folder structure within which all the results are saved is explained.
\end{enumerate}

\section{Future work}

\subsection{Test on more recordings}

In this project, only recording 4 from the 25 available was used. This was due to the fact that it was the dataset with the lowest number of neurons. An extension of this project could investigate the structure of the rest of recordings in the dataset. 

\subsection{Increase the complexity of the network model}

The model of the network employed in this project consisted of only excitatory neurons. However, biological neural networks consist of many different kinds of neurons. Future work could study the effect of having inhibitory neurons in the system but inferring the connectivity of the network with the assumption of them all being excitatory. 

\subsection{Random spiking cluster model}

The network model used in this project was the random spikes model. This was a very simple introductory method to mimicking biological neural networks. However, a more complex model could be studied where clusters of neurons spike at random times. Otherwise, due to the large differences yet to be solved between the real recordings and simulated network readings, more parameters can be changed so that the similarity is increased.

\subsection{Update the spiking prediction metric}

A method for evaluating the performance of NetRate on networks without a ground truth was proposed and tested in chapter 5. This is a very simple metric, however, it fails to make the best predictions possible. For this reason, an update to this metric could be proposed where not only the first neuron to spike is given, but also all the previous spikes in the recording. With such information it could be possible to estimate the membrane potential of all the neurons in the network and obtain a higher prediction score.


\input{appendix/appendix.tex}

\printbibliography 
% \bibliographystyle{unsrt}
% \bibliography{bibs/sample}
% \addcontentsline{toc}{chapter}{Bibliography}

\end{document}
